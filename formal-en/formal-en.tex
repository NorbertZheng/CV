\documentclass{resume}
\usepackage{setspace}
\renewcommand{\baselinestretch}{0.8}
\begin{document}

\name{\textbf{HUI(NORBERT) ZHENG}}

\contactinfo{
  Phone: +86 137-9231-5475}{Email: \href{mailto:fassial19991217@gmail.com}{fassial19991217@gmail.com}\\
  \href{https://fassial.github.io}{https://fassial.github.io}
}{}{}

%% research interests
\section{{\bfseries RESEARCH INTERESTS}}
\begin{itemize}[parsep=0.2ex]
  \item \textbf{Neuroscience}: Computational Neuroscience; Machine Learning; Brain Decoding
\end{itemize}

%% education
\section{{\bfseries EDUCATION}}
\datedsubsection{{\bfseries Peking University, Beijing, China}}{}{Sept. 2021 - Present}
\datedsubsection{}{\textit{Ph.D. in Biology}}{}
\begin{itemize}[parsep=0.1ex]
  \item \textbf{Advisor}: Dr. Yunzhe Liu
\end{itemize}
\datedsubsection{{\bfseries Wuhan University, Wuhan, China}}{}{Sept. 2017 - Jun. 2021}
\datedsubsection{}{\textit{B.S in Computer Science  (with honors)}}{}
\begin{itemize}[parsep=0.1ex]
  \item \textbf{GPA}: 3.84/4.00(92.1/100)
  \item \textbf{Rank}: 2/32(selected from 363 students in School of Computer Science, Wuhan University)
  \item \textbf{Exchange}: Visiting Student at University of California, Berkeley(2019 summer)
\end{itemize}

%% publications
\section{{\bfseries PUBLICATIONS}}
\flushleft{* indicates equal contribution.}
\begin{itemize}[parsep=0.2ex]
  % 2021
  \item Wenquan Xu, Haoyu Song, Linyang Hou, \textbf{Hui Zheng}, Xinggong Zhang, Chuwen Zhang, Wei Hu, Yi Wang, Bin Liu, "SODA: Similar 3D Object Detection Accelerator at Network Edge for Autonomous Driving," in IEEE International Conference on Computer Communications (INFOCOM), 2021.
  % 2020
  \item Yunzhe Li*, \textbf{Hui Zheng*}, He Zhu*, Haojun Ai and Xiaowei Dong, "Cross-People Mobile-Phone Based Airwriting Character Recognition," in International Conference on Pattern Recognition (ICPR), 2020.
\end{itemize}

%% research experience
\section{{\bfseries RESEARCH EXPERIENCE}}
% pku-yunzhe liu
\datedsubsection{\textbf{Peking University, Beijing, China}}{}{Jul. 2023 - Present}
\datedsubsection{}{Research Assistant}{}
\datedsubsection{}{Advisor: Dr. Yunzhe Liu}{}
\begin{itemize}[parsep=0.2ex]
  \item Research the decoding problem of MEG-image paradigm.
  \item Build a pre-train model that leverages MEG data across \~400 subjects to get a good finetune prior, then use CLIP supervision to get a high-performance classifier to provide tools for cognitive neuroscience experiments.
\end{itemize}
% pku-yunzhe liu
\datedsubsection{\textbf{Peking University, Beijing, China}}{}{Jan. 2023 - Present}
\datedsubsection{}{Research Assistant}{}
\datedsubsection{}{Advisor: Dr. Yunzhe Liu}{}
\begin{itemize}[parsep=0.2ex]
  \item Research the decoding problem of sEEG-speak paradigm.
  \item Build a naive machine learning model that decodes sentences from sEEG signals.
\end{itemize}
% pku-yunzhe liu
\datedsubsection{\textbf{Peking University, Beijing, China}}{}{Jan. 2023 - Present}
\datedsubsection{}{Research Assistant}{}
\datedsubsection{}{Advisor: Dr. Yunzhe Liu}{}
\begin{itemize}[parsep=0.2ex]
  \item Research the decoding problem of EEG-sleep under TMR (Target Memory Reactivation) paradigm.
  \item Build a contrastive learning model that leverages both awake and sleep data across \~60 subjects to decode sleep from EEG signals to provide tools for cognitive neuroscience experiments.
\end{itemize}
% pku-yunzhe liu
\datedsubsection{\textbf{Peking University, Beijing, China}}{}{Jan. 2022 - Present}
\datedsubsection{}{Research Assistant}{}
\datedsubsection{}{Advisor: Dr. Yunzhe Liu}{}
\begin{itemize}[parsep=0.2ex]
  \item Research the role of replay for memory consolidation.
  \item Build a machine replicate the replay observed in both human and rodent.
\end{itemize}
% pku-si wu
\datedsubsection{\textbf{Peking University, Beijing, China}}{}{Mar. 2021 - Aug. 2021}
\datedsubsection{}{Research Assistant}{}
\datedsubsection{}{Advisor: Dr. Si Wu}{}
\begin{itemize}[parsep=0.2ex]
  \item Researched the global to local information processing.
  \item Analyzed the dynamics of the neural network model with gap junction.
\end{itemize}
% nibs-minmin luo
\datedsubsection{\textbf{Tsinghua University, Beijing, China}}{}{Sept. 2020 - Mar. 2021}
\datedsubsection{}{Research Assistant}{}
\datedsubsection{}{Advisor: Dr. Minmin Luo}{}
\begin{itemize}[parsep=0.2ex]
  \item Researched the role of pituitary cells in systemic neuroinflammation at the single-cell transcriptome level.
  \item Revealed the transcriptional differences of different types of pituitary cells in the process of central nervous endocrine inflammation regulation.
\end{itemize}
% thu-bin liu
\datedsubsection{\textbf{Tsinghua University, Beijing, China}}{}{May. 2020 - Aug. 2020}
\datedsubsection{}{Research Assistant}{}
\datedsubsection{}{Advisor: Dr. Bin Liu}{}
\begin{itemize}[parsep=0.2ex]
  \item Researched the real-time processing of autonomous driving in the Internet of Vehicles.
  \item Designed efficient algorithms for the novel TCAM-NMC in-network accelerator, which accelerates the MEC-assisted similar 3D object detection for autonomous driving.
\end{itemize}
% whu-haojun ai
\datedsubsection{\textbf{Wuhan University, Wuhan, China}}{}{Feb. 2020 - Apr. 2020}
\datedsubsection{}{Research Assistant}{}
\datedsubsection{}{Advisor: Dr. Haojun Ai}{}
\begin{itemize}[parsep=0.2ex]
  \item Researched transfer learning in Air-Writing.
  \item Developed a system that could transfer between different people.
\end{itemize}
% whu-shubo liu
%\datedsubsection{\textbf{Wuhan University, Wuhan, China}}{}{}
%\datedsubsection{}{Research Assistant}{}
%\datedsubsection{}{Advisor: Dr. Shubo Liu} and Dr. Zhaohui Cai}{May. 2019 - Jan. 2020}

%% teaching expr
%\section{{\bfseries TEACHING EXPERIENCE}}
%\begin{itemize}[parsep=0.2ex]
%\end{itemize}

%% awards & honors
\section{{\bfseries AWARDS AND HONORS}}
% 2021
\datedsubsection{Outstanding Graduate \textbf{(12 out of 127, 10\%)}, Wuhan University}{}{Apr. 2021}
% 2020
\datedsubsection{National Scholarship, Wuhan University}{}{Oct. 2020}
\datedsubsection{Excellent Student Scholarship \textbf{(Rank: 1/32)}, Wuhan University}{}{Oct. 2020}
% 2019
\datedsubsection{National Second Prize of FPGA Innovation Design Competition, China}{}{Dec. 2019}
\datedsubsection{National Second Prize of Intelligent Robot Fighting Competition, China}{}{Oct. 2019}
\datedsubsection{Excellent Student Scholarship \textbf{(Rank: 4/32)}, Wuhan University}{}{Oct. 2019}
% 2018
\datedsubsection{Excellent Student Scholarship \textbf{(Rank: 8/32)}, Wuhan University}{}{Oct. 2018}
% 2017
\datedsubsection{Freshman Scholarship, Wuhan University}{}{Oct. 2017}

%%% projects
%\section{{\bfseries Projects}}
%% whu-airwriting
%\datedsubsection{\href{https://github.com/Fassial/NUFIC2019-WHU}{\textbf{Air-Writing Recognition based on Deep Learning}}, Team Leader}{}{}
%\datedsubsection{}{School of Cyber Science and Engineering, Wuhan University}{}
%\datedsubsection{}{Advised by \href{http://cse.whu.edu.cn/index.php?s=/home/szdw/detail/id/95.html}{A/Prof. Haojun Ai}}{Oct. 2019 - Dec. 2019}
%\begin{itemize}[parsep=0.2ex]
%  \item Works of \href{http://fpga.icisc.cn/}{FPGA Innovation Design Competition}. Use Bluetooth ring for Air-Writing. A more natural way of Human-Computer Interaction.
%  \item Use FPGA to filter and unpack the data collected by the acceleration sensor. Then transfer the data to the embedded Arm and use the deep neural network for prediction.
%\end{itemize}
%% whu-mips
%\datedsubsection{\href{https://github.com/trifling-mips}{\textbf{2-issue MIPS-CPU}}, Team Leader}{}{}
%\datedsubsection{}{School of Computer Science, Wuhan University}{}
%\datedsubsection{}{Co-advised by \href{http://cs.whu.edu.cn/teacherinfo.aspx?id=309}{Prof. Shubo Liu} and \href{http://cs.whu.edu.cn/teacherinfo.aspx?id=266}{A/Prof. Zhaohui Cai}}{Jan. 2020 - Aug. 2020}
%\begin{itemize}[parsep=0.2ex]
%  \item Works of \href{http://www.nscscc.org/}{NSCSCC2020}. Use MIPS32 ISA. Run at 80MHz.
%  \item Support all instructions necessary to start the linux kernel. Parallelize TLB and cache. Reach 20 points in the NSCSCC performance test.
%\end{itemize}
%% whu-lcore
%\datedsubsection{\href{https://github.com/Fassial/Lcore}{\textbf{Lcore}}, Team Leader}{}{}
%\datedsubsection{}{School of Computer Science, Wuhan University}{}
%\datedsubsection{}{Co-advised by \href{http://cs.whu.edu.cn/teacherinfo.aspx?id=309}{Prof. Shubo Liu} and \href{http://cs.whu.edu.cn/teacherinfo.aspx?id=266}{A/Prof. Zhaohui Cai}}{Aug. 2019 - Sept. 2019}
%\begin{itemize}[parsep=0.2ex]
%  \item Works of \href{http://www.nscscc.org/}{NSCSCC2019}. A simple operating system, running on MIPS-CPU.
%  \item Support basic process switching, memory management and shell interaction, etc.
%\end{itemize}
%% whu-oddb
%\datedsubsection{\href{https://github.com/Fassial/ODDB-Lab}{\textbf{Object-Deputy DataBase}}, Team Member}{}{}
%\datedsubsection{}{School of Computer Science, Wuhan University}{}
%\datedsubsection{}{Advised by \href{http://cs.whu.edu.cn/teacherinfo.aspx?id=297}{Prof. Zhiyong Peng}}{Jan. 2020 - Mar. 2020}
%\begin{itemize}[parsep=0.2ex]
%  \item The design of Database Design and Implementation Course.
%  \item Realize basic operations of ODDB, such as adding, deleting, modifying and searching.
%\end{itemize}

%% skills
%\section{{\bfseries Skills}}
%\begin{itemize}[parsep=0.2ex]
%  \item \textbf{Programming}: systemVerilog, C, python, R, java, LaTex, javascript, matlab
%  \item \textbf{Development Framework}: pytorch, tensorflow, vue
%  \item \textbf{English Level}: \textbf{CET-4} (538), \textbf{CET-6} (533)
%\end{itemize}

%%% leadership
%\section{{\bfseries Leadership}}
%% hongyi honor college
%\datedsubsection{\textbf{Minister of Network Technology Department of Student Union}}{}{}
%\datedsubsection{}{Hongyi Honor College, Wuhan University}{Sept. 2018 - Jun. 2019}
%% WHU-MSC
%\datedsubsection{\textbf{Vice-Chairman of Microsoft Student Club}}{}{}
%\datedsubsection{}{Wuhan University}{Sept. 2019 - Jun. 2020}

\end{document}

\documentclass{resume}
\usepackage{setspace}
\renewcommand{\baselinestretch}{0.8}
\begin{document}

\name{\textbf{郑晖}}
    
\contactinfo{
  Phone: +86 137-9231-5475}{Email: \href{mailto:fassial19991217@gmail.com}{fassial19991217@gmail.com}\\
  \href{https://norbertzheng.github.io}{https://norbertzheng.github.io}
}{}{}

\section{{\bfseries 研究兴趣}}
\begin{itemize}[parsep=0.2ex]
\item \textbf{机器学习}:数据挖掘;脑机接口;计算神经科学
\end{itemize}

\section{{\bfseries 教育经历}}
\datedsubsection{{\bfseries 北京大学}}{}{北京,中国}
\datedsubsection{}{博士,神经科学,前沿交叉学科研究院}{2021 年 9 月 至今}
\begin{itemize}[parsep=0.1ex]
  \item \textbf{导师}:柳昀哲
\end{itemize}
\datedsubsection{{\bfseries 武汉大学}}{}{武汉,中国}
\datedsubsection{}{学士,计算机科学与技术,弘毅学堂}{2017 年 9 月 - 2021 年 6 月}
\begin{itemize}[parsep=0.1ex]
  \item \textbf{GPA}:3.84/4.00(92.1/100)
  \item \textbf{专业排名}:2/32(从武汉大学计算机学院的363名学生中选拔出来)
  \item \textbf{交换经历}:加州大学伯克利分校访问(2019年暑假)
\end{itemize}

\section{{\bfseries 论文}}
\begin{itemize}[parsep=0.2ex]
  % 2024
  \item \textbf{Hui Zheng*}, Haiteng Wang*, Weibang Jiang, Zhongtao Chen, Li He, Peiyang Lin, Penghu Wei, Guoguang Zhao and Yunzhe Liu, "Du-IN: Discrete units-guided mask modeling for decoding speech from Intracranial Neural Signals," in submission.
  \item \textbf{Hui Zheng*}, Zhongtao Chen*, Haiteng Wang, Jianyang Zhou, Lin Zheng, Peiyang Lin and Yunzhe Liu, "SI-SD: Sleep Interpreter through awake-guided cross-subject Semantic Decoding," in submission.
  % 2021
  \item Wenquan Xu, Haoyu Song, Linyang Hou, \textbf{Hui Zheng}, Xinggong Zhang, Chuwen Zhang, Wei Hu, Yi Wang, Bin Liu, "SODA: Similar 3D Object Detection Accelerator at Network Edge for Autonomous Driving," in IEEE International Conference on Computer Communications (INFOCOM), 2021.
  % 2020
  \item Yunzhe Li*, \textbf{Hui Zheng*}, He Zhu*, Haojun Ai and Xiaowei Dong, "Cross-People Mobile-Phone Based Airwriting Character Recognition," in International Conference on Pattern Recognition (ICPR), 2020.
\end{itemize}

%% research experience
\section{{\bfseries 科研经历}}
% pku-yunzhe liu
\datedsubsection{\textbf{A Neural Network Model based on Tolman-Eichenbaum Machine to explain Human Replay},科研实习}{}{}
\datedsubsection{}{前沿交叉学科研究院,北京大学}{}
\datedsubsection{}{导师:\href{https://cibr.ac.cn/science/team/detail/763?language=cn}{柳昀哲教授}}{2022 年 1 月 - 2022 年 12 月}
\begin{itemize}[parsep=0.2ex]
  \item 研究重播对于记忆巩固的作用。
  \item 建造一个模型来解释在人类和啮齿动物中观察到的神经重放现象。
\end{itemize}
% pku-si wu
\datedsubsection{\textbf{A Neural Network Model with Gap Junction for Global Feature Extraction},科研实习}{}{}
\datedsubsection{}{前沿交叉学科研究院,北京大学}{}
\datedsubsection{}{导师:\href{http://www.aais.pku.edu.cn/duiwu/showproduct.php?id=208}{吴思教授}}{2021 年 3 月 - 2021 年 8 月}
\begin{itemize}[parsep=0.2ex]
  \item 研究人类大脑中视觉通路全局到局部的信息处理。
  \item 分析了间隙连接神经网络模型的动力学。
\end{itemize}
% thu-bin liu
\datedsubsection{\textbf{SODA: Similar 3D Object Detection Accelerator at Network Edge for Autonomous Driving},科研实习}{}{}
\datedsubsection{}{计算机科学与技术系,清华大学}{}
\datedsubsection{}{导师:\href{http://www.cs.tsinghua.edu.cn/publish/cs/4616/2013/20130424093153561198286/20130424093153561198286_.html}{刘斌教授}}{2020 年 5 月 - 2020 年 8 月}
\begin{itemize}[parsep=0.2ex]
  \item 研究与车联网中的自动驾驶实时处理有关的问题。
  \item SODA为自动驾驶问题加速了MEC辅助的类似3D对象检测的过程。我们为新型TCAM-NMC网络加速器设计了有效的算法,并通过广泛的评估,确认了该架构在自动驾驶问题上的可行性和性能优势。
\end{itemize}
% whu-haojun ai
\datedsubsection{\textbf{Cross-People Mobile-Phone Based Airwriting Character Recognition},科研实习}{}{}
\datedsubsection{}{国家网络安全学院,武汉大学}{}
\datedsubsection{}{导师:\href{http://cse.whu.edu.cn/index.php?s=/home/szdw/detail/id/95.html}{艾浩军副教授}}{2020 年 2 月 - 2020 年 4 月}
\begin{itemize}[parsep=0.2ex]
  \item 研究与空中手写字符识别中迁移学习有关的问题。
  \item 我们开发了可以在不同人之间迁移的系统。该系统具有更好的个性化识别性能。
\end{itemize}

%% awards & scholarships
\section{{\bfseries 荣誉 \& 奖学金}}
% 2021
\datedsubsection{优秀毕业生 \textbf{(10\%)},武汉大学}{}{2021 年 4 月}
% 2020
\datedsubsection{国家奖学金,武汉大学}{}{2020 年 10 月}
\datedsubsection{优秀学生奖学金 \textbf{(排名:1/32)},武汉大学}{}{2020 年 10 月}
% 2019
\datedsubsection{全国FPGA创新设计大赛二等奖,中国}{}{2019 年 12 月}
\datedsubsection{中国智能机器人格斗大赛二等奖,中国}{}{2019 年 10 月}
\datedsubsection{优秀学生奖学金 \textbf{(排名:4/32)},武汉大学}{}{2019 年 10 月}
% 2018
\datedsubsection{优秀学生奖学金 \textbf{(排名:8/32)},武汉大学}{}{2018 年 10 月}
% 2017
\datedsubsection{新生奖学金,武汉大学}{}{2017 年 10 月}

\end{document}

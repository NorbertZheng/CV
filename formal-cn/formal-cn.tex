\documentclass{resume}
\usepackage{setspace}
\renewcommand{\baselinestretch}{0.8}
\begin{document}

\name{\textbf{\href{https://fassial.github.io}{郑晖}}}
    
\contactinfo{
  Phone: +86 137-9231-5475}{Email: \href{mailto:fassial19991217@gmail.com}{fassial19991217@gmail.com}\\
  \href{https://fassial.github.io}{https://fassial.github.io}
}{}{}

\section{{\bfseries 研究兴趣}}
\begin{itemize}[parsep=0.2ex]
\item \textbf{神经科学}:计算神经科学;机器学习;脑机接口
\end{itemize}

\section{{\bfseries 教育经历}}
\datedsubsection{{\bfseries 北京大学}}{}{北京,中国}
\datedsubsection{}{博士,生物学,前沿交叉学科研究院}{2021 年 9 月 - 2026 年 6 月(expected)}
\begin{itemize}[parsep=0.1ex]
    \item \textbf{导师}:柳昀哲
\end{itemize}
\datedsubsection{{\bfseries 武汉大学}}{}{武汉,中国}
\datedsubsection{}{学士,计算机科学与技术,弘毅学堂}{2017 年 9 月 - 2021 年 6 月}
\begin{itemize}[parsep=0.1ex]
    \item \textbf{GPA}:3.84/4.00(92.1/100)
    \item \textbf{专业排名}:2/32(从武汉大学计算机学院的363名学生中选拔出来)
    \item \textbf{交换经历}:加州大学伯克利分校访问(2019年暑假)
\end{itemize}

\section{{\bfseries 论文}}
\begin{itemize}[parsep=0.2ex]
    \item Yunzhe Li*, \textbf{Hui Zheng*}, He Zhu*, Haojun Ai and Xiaowei Dong. "Cross-People Mobile-Phone Based Airwriting Character Recognition". ICPR2020 Accepted.
    \item Wenquan Xu, Haoyu Song, Linyang Hou, \textbf{Hui Zheng}, Xinggong Zhang, Chuwen Zhang, Wei Hu, Yi Wang, Bin Liu. "SODA: Similar 3D Object Detection Accelerator at Network Edge for Autonomous Driving". INFOCOM2021 Accepted.
\end{itemize}

%% research experience
\section{{\bfseries 科研经历}}
% pku-yunzhe liu
\datedsubsection{\textbf{A Naive Machine Learning Model to Decode Speak from sEEG signals},科研实习}{}{}
\datedsubsection{}{前沿交叉学科研究院,北京大学}{}
\datedsubsection{}{导师:\href{https://cibr.ac.cn/science/team/detail/763?language=cn}{柳昀哲教授}}{2023 年 1 月至今}
\begin{itemize}[parsep=0.2ex]
    \item 研究sEEG-speak范式的解码问题。
    \item 构建一个简单的机器学习模型,从 sEEG 信号中解码句子。
\end{itemize}
% pku-yunzhe liu
\datedsubsection{\textbf{A Contrastive Learning model to Decode Sleep from EEG signals},科研实习}{}{}
\datedsubsection{}{前沿交叉学科研究院,北京大学}{}
\datedsubsection{}{导师:\href{https://cibr.ac.cn/science/team/detail/763?language=cn}{柳昀哲教授}}{2023 年 1 月至今}
\begin{itemize}[parsep=0.2ex]
    \item 研究TMR(目标记忆重新激活)范式下脑电睡眠的解码问题。
    \item 建立一个对比学习模型,利用约 60 名受试者的清醒和睡眠数据,从脑电图信号中解码睡眠,为认知神经科学实验提供工具。
\end{itemize}
% pku-yunzhe liu
\datedsubsection{\textbf{A Neural Network Model based on Tolman-Eichenbaum Machine to explain Human Replay},科研实习}{}{}
\datedsubsection{}{前沿交叉学科研究院,北京大学}{}
\datedsubsection{}{导师:\href{https://cibr.ac.cn/science/team/detail/763?language=cn}{柳昀哲教授}}{2022 年 1 月至今}
\begin{itemize}[parsep=0.2ex]
    \item 研究重播对于记忆巩固的作用。
    \item 建造一个模型来解释在人类和啮齿动物中观察到的神经重放现象。
\end{itemize}
% pku-si wu
\datedsubsection{\textbf{A Neural Network Model with Gap Junction for Global Feature Extraction},科研实习}{}{}
\datedsubsection{}{前沿交叉学科研究院,北京大学}{}
\datedsubsection{}{导师:\href{http://www.aais.pku.edu.cn/duiwu/showproduct.php?id=208}{吴思教授}}{2021 年 3 月 - 2021 年 8 月}
\begin{itemize}[parsep=0.2ex]
    \item 研究人类大脑中视觉通路全局到局部的信息处理。
    \item 分析了间隙连接神经网络模型的动力学。
\end{itemize}
% nibs-minmin luo
\datedsubsection{\textbf{Single-Cell Transcriptomics to uncover the Relationships between Inflammation and Hormone in Pituitary Cells},科研实习}{}{}
\datedsubsection{}{北京生命科学研究所,中国}{}
\datedsubsection{}{导师:\href{http://www.nibs.ac.cn/yjsjyimgshow.php?cid=5&sid=6&id=775}{罗敏敏教授}}{2020 年 9 月 - 2021 年 3 月}
\begin{itemize}[parsep=0.2ex]
    \item 在单细胞转录组水平上研究与垂体细胞在系统性神经炎症中作用有关的问题。
    \item 我们揭示了中枢神经内分泌炎症调节过程中不同类型垂体细胞的转录差异。我们发现了一组在不同类型的垂体细胞中均匀表达的转录因子。
\end{itemize}
% thu-bin liu
\datedsubsection{\textbf{SODA: Similar 3D Object Detection Accelerator at Network Edge for Autonomous Driving},科研实习}{}{}
\datedsubsection{}{计算机科学与技术系,清华大学}{}
\datedsubsection{}{导师:\href{http://www.cs.tsinghua.edu.cn/publish/cs/4616/2013/20130424093153561198286/20130424093153561198286_.html}{刘斌教授}}{2020 年 5 月 - 2020 年 8 月}
\begin{itemize}[parsep=0.2ex]
    \item 研究与车联网中的自动驾驶实时处理有关的问题。
    \item SODA为自动驾驶问题加速了MEC辅助的类似3D对象检测的过程。我们为新型TCAM-NMC网络加速器设计了有效的算法,并通过广泛的评估,确认了该架构在自动驾驶问题上的可行性和性能优势。
\end{itemize}
% whu-haojun ai
\datedsubsection{\textbf{Cross-People Mobile-Phone Based Airwriting Character Recognition},科研实习}{}{}
\datedsubsection{}{国家网络安全学院,武汉大学}{}
\datedsubsection{}{导师:\href{http://cse.whu.edu.cn/index.php?s=/home/szdw/detail/id/95.html}{艾浩军副教授}}{2020 年 2 月 - 2020 年 4 月}
\begin{itemize}[parsep=0.2ex]
    \item 研究与空中手写字符识别中迁移学习有关的问题。
    \item 我们开发了可以在不同人之间迁移的系统。该系统具有更好的个性化识别性能。
\end{itemize}
% whu-shubo liu
\datedsubsection{\textbf{RISC-V Super Scalar Processor Design and Internet of Things Application},科研实习}{}{}
\datedsubsection{}{计算机学院,武汉大学}{}
\datedsubsection{}{导师:\href{http://cs.whu.edu.cn/teacherinfo.aspx?id=309}{刘树波教授} \& \href{http://cs.whu.edu.cn/teacherinfo.aspx?id=266}{蔡朝晖副教授}}{2019 年 5 月 - 2020 年 1 月}

%% projects
\section{{\bfseries 项目}}
% whu-airwriting
\datedsubsection{\href{https://github.com/Fassial/NUFIC2019-WHU}{\textbf{Air-Writing Recognition based on Deep Learning}},团队队长}{}{}
\datedsubsection{}{国家网络安全学院,武汉大学}{}
\datedsubsection{}{导师:\href{http://cse.whu.edu.cn/index.php?s=/home/szdw/detail/id/95.html}{艾浩军副教授}}{2019 年 10 月 - 2019 年 12 月}
\begin{itemize}[parsep=0.2ex]
    \item \href{http://fpga.icisc.cn/}{全国FPGA创新设计大赛}比赛作品。使用蓝牙环进行空中手写字符识别,一种更自然的人机交互方式。
    \item 使用FPGA对加速度传感器收集的数据进行解包和滤波。然后将数据传输到嵌入式Arm核,使用深度神经网络进行预测。
\end{itemize}
% whu-mips
\datedsubsection{\href{https://github.com/trifling-mips}{\textbf{2-issue MIPS-CPU}},团队队长}{}{}
\datedsubsection{}{计算机学院,武汉大学}{}
\datedsubsection{}{导师:\href{http://cs.whu.edu.cn/teacherinfo.aspx?id=309}{刘树波教授} \& \href{http://cs.whu.edu.cn/teacherinfo.aspx?id=266}{蔡朝晖副教授}}{2020 年 1 月 - 2020 年 8 月}
\begin{itemize}[parsep=0.2ex]
    \item \href{http://www.nscscc.org/}{全国大学生系统能力大赛}比赛作品。使用MIPS32指令集,以80MHz主频运行。
    \item 支持启动linux内核所需的所有指令,并行化TLB和Cache。在NSCSCC性能测试中达到20分。
\end{itemize}
% whu-lcore
\datedsubsection{\href{https://github.com/Fassial/Lcore}{\textbf{Lcore}},团队队长}{}{}
\datedsubsection{}{计算机学院,武汉大学}{}
\datedsubsection{}{导师:\href{http://cs.whu.edu.cn/teacherinfo.aspx?id=309}{刘树波教授} \& \href{http://cs.whu.edu.cn/teacherinfo.aspx?id=266}{蔡朝晖副教授}}{2019 年 8 月 - 2019 年 9 月}
\begin{itemize}[parsep=0.2ex]
    \item \href{http://www.nscscc.org/}{全国大学生系统能力大赛}比赛作品。一个在MIPS-CPU上运行的简单操作系统。
    \item 支持基本的进程切换、内存管理和Shell交互等。
\end{itemize}
% whu-oddb
\datedsubsection{\href{https://github.com/Fassial/ODDB-Lab}{\textbf{Object-Deputy DataBase}},团队成员}{}{}
\datedsubsection{}{计算机学院,武汉大学}{}
\datedsubsection{}{导师:\href{http://cs.whu.edu.cn/teacherinfo.aspx?id=297}{彭智勇教授}}{2020 年 1 月 - 2020 年 3 月}
\begin{itemize}[parsep=0.2ex]
    \item 数据库设计与实现的课程设计。
    \item 实现ODDB的基本操作,例如添加、删除、修改和搜索。
\end{itemize}

%% awards & scholarships
\section{{\bfseries 荣誉 \& 奖学金}}
% 2021
\datedsubsection{优秀毕业生 \textbf{(10\%)},武汉大学}{}{2021 年 4 月}
% 2020
\datedsubsection{国家奖学金,武汉大学}{}{2020 年 10 月}
\datedsubsection{优秀学生奖学金 \textbf{(排名:1/32)},武汉大学}{}{2020 年 10 月}
% 2019
\datedsubsection{全国FPGA创新设计大赛二等奖,中国}{}{2019 年 12 月}
\datedsubsection{中国智能机器人格斗大赛二等奖,中国}{}{2019 年 10 月}
\datedsubsection{优秀学生奖学金 \textbf{(排名:4/32)},武汉大学}{}{2019 年 10 月}
% 2018
\datedsubsection{优秀学生奖学金 \textbf{(排名:8/32)},武汉大学}{}{2018 年 10 月}
% 2017
\datedsubsection{新生奖学金,武汉大学}{}{2017 年 10 月}

%% skills
\section{{\bfseries 技能}}
\begin{itemize}[parsep=0.2ex]
    \item \textbf{编程语言}: python, matlab, systemVerilog, C, R, java, LaTex, javascript
    \item \textbf{开发框架}: pytorch, tensorflow, vue
    \item \textbf{英语水平}: \textbf{CET-4} (538), \textbf{CET-6} (533)
\end{itemize}

%% leadership
\section{{\bfseries 领导能力}}
% hongyi honor college
\datedsubsection{\textbf{院学生会网络技术部部长}}{}{}
\datedsubsection{}{弘毅学堂,武汉大学}{2018 年 9 月 - 2019 年 6 月}
% WHU-MSC
\datedsubsection{\textbf{微软学生俱乐部副主席}}{}{}
\datedsubsection{}{武汉大学}{2019 年 9 月 - 2020 年 6 月}

\end{document}
